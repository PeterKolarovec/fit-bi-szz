\documentclass{report}
\usepackage[utf8]{inputenc}
\usepackage[T1]{fontenc}
\usepackage{hyperref}
\hypersetup{
    colorlinks=true,
    linkcolor=red,
    filecolor=magenta,      
    urlcolor=cyan,
}

\usepackage{graphicx}
\usepackage{subcaption}

\usepackage{amssymb}
\usepackage{amsthm}

\usepackage{todonotes}

% margins
\usepackage[a4paper, total={6.5in, 10in}]{geometry}

% spacing
\setlength{\parindent}{0em}
\setlength{\parskip}{1em}

% fonts (the same as NIPS 2016)
% \renewcommand{\rmdefault}{ptm}
% \renewcommand{\sfdefault}{phv}

% citations
\usepackage[style=iso-numeric,backend=biber]{biblatex}
\addbibresource{bibliography.bib}

\newtheoremstyle{myStyle}
  {20pt} % Space above
  {20pt} % Space below
  {} % Body font
  {} % Indent amount
  {\bfseries} % Theorem head font
  {:} % Punctuation after theorem head
  {.5em} % Space after theorem head
  {} % Theorem head spec (can be left empty, meaning `normal')

\theoremstyle{myStyle}
\newtheorem{definition}{Definice}[subsection]

\theoremstyle{myStyle}
\newtheorem{theorem}{Věta}[subsection]

% -----------------------------------------------------------------------------

\title{Přehled matematických vět a definic.\vspace{-1em}}
% \author{Matej Choma}
\date{\vspace{-1em}\today}

\begin{document}

\maketitle

% #################################################################################################
\chapter{BI-LIN, Lineární algebra}
% #################################################################################################

% =========================================================
\section{Background}

% =========================================================
\section{12 -- Soustavy lineárních rovnic: Frobeniova věta a související pojmy, vlastnosti a popis množiny řešení, Gaussova eliminační metoda.}

% =========================================================
\section{13 -- Matice: součin matic, regulární matice, inverzní matice a její výpočet, vlastní čísla matice a jejich výpočet, diagonalizace matice.}


% #################################################################################################
\chapter{BI-MLO, Matematická logika}
% #################################################################################################

% =========================================================
\section{Background}

% =========================================================
\section{14 -- Výroková logika: syntax a sémantika výrokových formulí, pravdivostní ohodnocení, logický důsledek, ekvivalence a jejich zjišťování. Universální systém logických spojek, disjunktivní a konjunktivní normální tvary, úplné a minimální tvary.}

% =========================================================
\section{15 -- Predikátová logika: jazyk, interpretace, pravdivost formulí, logický důsledek a ekvivalence. Formalizace matematických tvrzení a jejich negace. Teorie a jejich modely (např. uspořádání).}


% #################################################################################################
\chapter{BI-PST, Pravděpodobnost a statistika}
% #################################################################################################

% =========================================================
\section{Background}

% =========================================================
\section{25 -- Pravidla pro výpočty pravděpodobností, Bayesův vzorec. Náhodné veličiny, příklady rozdělení, distribuční funkce, hustota, momenty. Nezávislost náhodných jevů a veličin. Centrální limitní věta, zákony velkých čísel.}

% =========================================================
\section{26 -- Základy statistické indukce, náhodný výběr, bodové odhady pro střední hodnotu a rozptyl, intervalové odhady pro střední hodnotu, testování statistických hypotéz o střední hodnotě.}


% #################################################################################################
\chapter{BI-ZDM, Základy diskrétní matematiky}
% #################################################################################################

% =========================================================
\section{Background}

% =========================================================
\section{32 -- Metody řešení rekurentních rovnic, sestavování a řešení rekurentních rovnic při analýze časové složitosti algoritmů.}

% =========================================================
\section{33 -- Modulární aritmetika, základy teorie čísel, Malá Fermatova věta, diofantické rovnice, lineární kongruence, Čínská věta o zbytcích.}

% \clearpage

% #################################################################################################
\chapter{BI-ZMA, Základy matematické analýzy}
% #################################################################################################

% =========================================================
\section{Background}

\theorem[Může se hodit] Pro $a,b\in\mathbb{R}$ platí
$$
	a^{n+1}-b^{n+1}=(a-b)\sum_{k=0}^n a^{n-k}b^k.
$$

\definition[Posloupnost]\label{def:posloupnost} Zobrazení množiny $\mathbb{N}$ do množiny $\mathbb{R}$ nazýváme \textbf{reálná posloupnost}. Zapisujeme $(a_n)_{n=1}^{\infty}$.

\definition[Limita posloupnosti] Reálná posloupnost $(a_n)_{n=1}^{\infty}$ má \textbf{limitu} $\alpha\in\overline{\mathbb{R}}$, právě když pro každé okolí $H_{\alpha}$ bodu $\alpha$ lze nalézt $n_0 \in \mathbb{N}$ takové, že pro všechna $n\in\mathbb{N}$ větší než $n_0$ platí $a_n\in H_{\alpha}$. V symbolech
$$
	(\forall H_{\alpha})(\exists n_0 \in \mathbb{N})(\forall n\in\mathbb{N})(n>n_0 \Rightarrow a_n \in H_{\alpha}).
$$
Značí se $\lim\limits_{n\rightarrow\infty} a_n=\alpha$ a bez pojmu okolí můžeme přepsat na
$$
	(\forall \varepsilon\in\mathbb{R},\varepsilon >0)(\exists n_0 \in \mathbb{N})(\forall n\in\mathbb{N})(n>n_0 \Rightarrow |a_n-\alpha| < \varepsilon).
$$

\definition[Konvergence posloupnosti]\label{def:konvergence} Buď $(a_n )_{n=1}^{\infty}$ posloupnost. Pokud pro její limitu platí $\lim\limits_{n\rightarrow\infty} a_n \in\mathbb {R}$, pak se nazývá \textbf{konvergentní}. V ostatních případech ji nazýváme \textbf{divergentní}.

% =========================================================
\section{34 -- Limita a derivace funkce (definice a vlastnosti, geometrický význam), využití při vyšetřování průběhu funkce.}

\definition[Limita funkce]\label{def:limita} Buďte $f$ reálná funkce reálné proměnné a $a \in\overline{\mathbb{R}}$. Nechť $f$ je definovaná na okolí bodu $a$, s možnou výjimkou bodu $a$ samotného. Řekneme, že $c \in\overline{\mathbb{R}}$ je \textbf{limitou funkce $f$ v bodě $a$}, právě když pro každé okolí $H_c$ bodu $c$ existuje okolí $H_a$ bodu $a$ takové, že z podmínky
$$
	x\in H_a \setminus \{a\}
$$
plyne
$$
	f(x)\in H_c.
$$
V symboloch
$$
	(\forall H_c)(\exists H_a)(\forall x \in D_f)(x \in H_a \setminus \{a\} \Rightarrow f(x) \in H_c).
$$
Tuto skutečnost zapisujeme
$$
	\lim_{x\rightarrow a}f(x)=c,\quad \lim_a f=c.
$$

\definition[Limita funkce zleva/zprava] Buďte $f$ reálná funkce reálné proměnné a $a \in\mathbb {R}$. Nechť $f$ je definovaná na levém, resp. pravém, okolí bodu $a$. Řekneme, že $c \in\overline{\mathbb{R}}$ je \textbf{limitou funkce $f$ v bodě $a$ zleva, resp. zprava}, právě když
$$
	\lim_{x\rightarrow a_-}f(x)=c \Leftrightarrow (\forall H_c)(\exists H_a^{-})(\forall x \in D_f)(x \in H_a^{-} \setminus \{a\} \Rightarrow f(x) \in H_c),
$$
respektíve
$$
	\lim_{x\rightarrow a_+}f(x)=c \Leftrightarrow (\forall H_c)(\exists H_a^{+})(\forall x \in D_f)(x \in H_a^{+} \setminus \{a\} \Rightarrow f(x) \in H_c).
$$

% ---------------------------------------------------------
\subsection{Vlastnosti limit funkcí}
\theorem Nechť $a\in\mathbb{R}$. Limita $\lim\limits_{x\rightarrow a}f(x)$ existuje a je rovna $c\in\overline{\mathbb{R}}$, právě když existují obě jednostranné limity $\lim\limits_{x\rightarrow a_+}f(x)$ a $\lim\limits_{x\rightarrow a_-}f(x)$ a obě jsou rovny $c$. \hspace{.5cm}$\Rightarrow$\hspace{.5cm} Ak obě jednostranné limity funkce $f$ v bodě $a$ existují a jsou různé, alebo alespoň jedna z jednostranných limit funcke $f$ v bodě $a$ neexistuje, potom limita funkce $f$ v bodě $a$ neexistuje.

\theorem[Heine]\label{th:heine} $\lim\limits_{x\rightarrow a} f(x)=c$, právě když je $f$ definována na okolí bodu $a$ (s možnou výjimkou bodu $a$) a pro každou posloupnost $(x_n)_{n=1}^\infty$ s limitou $a$ a splňující
$$
	\{x_n|n\in\mathbb{N}\}\subset D_f\setminus\{a\}
$$
platí $\lim\limits_{n\rightarrow\infty} f(x_n)=c$. \hspace{.5cm}$\Rightarrow$\hspace{.5cm} Nechť $f$ je funkce definovaná na okolí bodu $a\in\overline{\mathbb{R}}$ a $(x_n)_{n=1}^{\infty},(z_n)_{n=1}^{\infty}$ jsou dvě reálné posloupnosti patřící do $D_f$, konvergující k $a$ a splňující podmínky $x_n\neq a$ a $z_n\neq a$ pro všechna $n\in\mathbb{N}$. Pokud limity
$$
	\lim_{n\rightarrow\infty}f(x_n)\quad\mathrm{a}\quad\lim_{n\rightarrow\infty}f(z_n)
$$
existují a jsou různé, nebo alespoň jedna z nich neexistuje, potom limita $\lim\limits_{x\rightarrow a}f(x)$ neexistuje.

\theorem Nechť $f$ a $g$ jsou funcke a $a\in\overline{\mathbb{R}}$. Potom
$$
	\lim_a (f+g)=\lim_a f + \lim_a g,
$$
$$
	\lim_a f\cdot g=\lim_a f \cdot \lim_a g,
$$
$$
	\lim_a \frac{f}{g}=\frac{\lim_a f}{\lim_a g},
$$
platí v případě, že výrazy na pravé strané jsou definovány a v posledním případě za předpokladu, že $\frac{f}{g}$ je definována na okolí bodu $a$ s možnou výjimkou bodu $a$ samotného.

\theorem[O limitě složené funkce] Nechť $f$ a $g$ jsou funkce, $a, b, c \in \overline{\mathbb{R}}$ a platí tři podmínky
\vspace{-5pt}
\begin{enumerate}
	\item $\lim\limits_{x\rightarrow a}g(x)=b$,
	\item $\lim\limits_{x\rightarrow b}f(x)=c$,
	\item buď $(\exists H_a)(\forall x \in D_g\cap H_a\setminus \{a\})(g(x)\neq b)$ nebo $(b\in D_f \mathrm{a} f(b)=c)$.
\end{enumerate}
\vspace{-5pt}
Potom platí $\lim\limits_{x\rightarrow a} f(g(x))=c$.

\theorem[Věta o limitě sevvřené funkce] Nechť pro funkce f, g, h a body platí:
\vspace{-5pt}
\begin{enumerate}
	\item $(\exists H_a)(\forall x \in H_a\setminus \{a\})(f(x)\leq g(x) \leq h(x)$,
	\item existují $\lim\limits_{x\rightarrow a}f(x)=\lim\limits_{x\rightarrow a}h(x)=c$.
\end{enumerate}
\vspace{-5pt}
Potom existuje i $\lim\limits_{x\rightarrow a}g(x)$ a je rovna $c$.

\theorem[l’Hospitalovo pravidlo] Nechť pro funkce $f$ a $g$ a bod $a \in\overline{\mathbb{R}}$ platí
\vspace{-5pt}
\begin{enumerate}
	\item $\lim\limits_{a}f = \lim\limits_{a}g =0$ nebo $\lim\limits_{a}|g| =+\infty$,
	\item existuje okolí $H_a$ bodu $a$ splňující $ H_a\setminus \{a\}\subset D_{f/g}\cap D_{f'/g'} $,
	\item existuje $\lim\limits_a \frac{f'}{g'}$.
\end{enumerate}
\vspace{-5pt}
Potom existuje $\lim\limits_a\frac{f}{g}$ a platí $\lim\limits_a\frac{f}{g} = \lim\limits_a\frac{f'}{g'}$.

% ---------------------------------------------------------
\subsection{Spojitost funkce}

\theorem[Spojitost funkce] Nechť $f$ je reálná funkce reálné proměnné a nechť bod $a\in D_f$. Řekneme, že funkce $f$ je \textbf{spojitá v bodě} $a$ jestliže nastáva alespoň jedna z nasledujících možností
\vspace{-5pt}
\begin{itemize}
	\item $\lim\limits_{x\rightarrow a}f(x)=f(a)$,
	\item $(\exists H_a)(H_a\cap D_f=H_a^+)$ a $\lim\limits_{x\rightarrow a+}f(x)=f(a)$,
	\item $(\exists H_a)(H_a\cap D_f=H_a^-)$ a $\lim\limits_{x\rightarrow a-}f(x)=f(a)$.
\end{itemize}
\vspace{-5pt}
Funkce $f$ je \textbf{spojitá v bodě $a$ zprava}, pokud $\lim\limits_{x\rightarrow a+}f(x)=f(a)$. Funkce $f$ je \textbf{spojitá v bodě $a$ zleva}, pokud $\lim\limits_{x\rightarrow a-}f(x)=f(a)$.

\theorem Funkce $f$ definovaná na okolí bodu $a\in D_f$ je spojitá v bodě $a$, právě když je spojitá v bodě $a$ zleva i zprava.

\textit{Keď to dáva zmysel, súčet, súčin, podiel a zlúčenie dvoch spojitých funkcií je opäť spojitá funkcia.}

% ---------------------------------------------------------
\subsection{Derivace funkce}

\definition[Derivace funkcie]\label{def:derivace} Nechť $f$ je funkce definovaná na okolí bodu $a\in\mathbb{R}$. Pokud existuje \hyperref[def:limita]{limita}
$$
	\lim_{x\rightarrow a}\frac{f(x)-f(a)}{x-a},\quad\mathrm{ekvivalentne}\quad\lim_{h\rightarrow 0}\frac{f(a+h)-f(a)}{h},
$$
nazveme její hodnotu \textbf{derivací funkce $f$ bodě $a$} a označíme $f'(a)$. Pokud je tato limita konečná (tj. $f'(a)\in\mathbb{R}$) řekneme, že funkce $f$ je \textbf{diferencovatelná} v bodě $a$.

\definition Buď $f$ funkce s definičním oborem $D_f$. Nechť $M$ označuje množinu všech $a \in D_f$ takových, že existuje konečná derivace $f'(a)$. \textbf{Derivací funkce} $f$ nazýváme funkci s definičním oborem $M$, která každému $x \in M$ přiřadí $f'(x)$. Tuto funkci značíme symbolem $f'$.

\definition Nechť existuje $f'(a)$. Tečnou funkce $f$ v bodě $a$ nazýváme
\vspace{-5pt}
\begin{itemize}
	\item přímku s rovnicí $x = a$ je-li funkce $f$ spojitá v bodě $a$ a $f'(a) = +\infty$ nebo $f' (a) =-\infty$.
	\item přímku s rovnicí $y = f(a) + f'(a)(x - a)$ je-li $f'(a) \in\mathbb{R}$ (tj. je-li $f$ je diferencovatelná v bodě $a$).
\end{itemize}

% ---------------------------------------------------------
\subsection{Vlastnosti derivace}

\theorem Je li funkce $f$ diferencovatelná v bodě $a$, pak je spojitá v bodě $a$.
$$
	f'(a)\in\mathbb{R} \quad\Rightarrow\quad \lim_{x\rightarrow a} f(x)=f(a)
$$

\theorem[Derivace součtu, součinu a podílu] Nechť funkce $f$ a $g$ jsou diferencovatelné v bodě $a$. Potom platí
\vspace{-5pt}
\begin{itemize}
	\item $(f+g)'(a) = f'(a) + g'(a)$
	\item $ (f\cdot g)'(a) = f'(a)g(a) + f(a)g'(a)$
	\item $\big( \frac{f}{g} \big)'(a) = \frac{f'(a)g(a)-f(a)g'(a)}{g(a)^2}$, pokud $g(a)\neq0$.
\end{itemize}
\vspace{-5pt}

\theorem[Derivace složené funkce] Nechť $g$ je funkce diferencovatelná v bodě $a$, $f$ je diferencovatelná v bodě $g(a)$. Potom funkce $f \circ g$ je diferencovatelná v bodě $a$ a platí
$$
	(f\circ g)'(a) = f'(g(a))\cdot g'(a).
$$

\textit{Videli ste niekedy deriváciu inverznej funkcie?}

% ---------------------------------------------------------
\subsection{Extrémy funkce}

\definition Řekneme, že funkce $f$ má v bodě $a \in D_f$
\vspace{-5pt}
\begin{itemize}
	\item\textbf{lokální minimum}$\quad\Leftrightarrow\quad (\exists H_a\subset D_f)(\forall x\in H_a)(f(x)\leq f(a)) $
	\item\textbf{lokální maximum}$\quad\Leftrightarrow\quad (\exists H_a\subset D_f)(\forall x\in H_a)(f(x)\geq f(a)) $
	\item\textbf{ostré lokální maximum}$\quad\Leftrightarrow\quad (\exists H_a\subset D_f)(\forall x\in H_a\setminus \{a\})(f(x)< f(a)) $
	\item\textbf{ostré lokální minimum}$\quad\Leftrightarrow\quad (\exists H_a\subset D_f)(\forall x\in H_a\setminus \{a\})(f(x)> f(a)) $
\end{itemize}
\vspace{-5pt}

\theorem[Nutná podmínka existence lokálního extrému] Nechť funkce $f$ má v bodě $a$ lokální extrém. Potom $f' (a) = 0$, nebo derivace v bodě $a$ neexistuje.

\theorem[Extrém spojité funkce na uzavřeném intervalu] Funkce $f$ spojitá a definovaná právě na uzavřeném intervalu $\langle a, b\rangle$ nabývá maxima a minima (tzv. \textbf{globální extrém}). Extrém může být pouze v krajních bodech $a, b$ a v bodech kde je derivace rovna 0 nebo neexistuje.

% ---------------------------------------------------------
\subsection{Vyšetřování průběhu funkce}

\theorem[Rolleova] Nechť funkce $f$ splňuje podmínky
\vspace{-5pt}
\begin{itemize}
	\item $f$ je spojitá na intervalu $\langle a,b \rangle$,
	\item $f$ má derivaci v každém bodě intervalu $(a,b)$,
	\item $f(a)=f(b)$.
\end{itemize}
\vspace{-5pt}
Potom existuje $c\in(a,b)$ tak, že $f'(c)=0$.

\theorem[Lagrangeova] Nechť funkce $f$ splňuje podmínky
\vspace{-5pt}
\begin{itemize}
	\item $f$ je spojitá na intervalu $\langle a,b \rangle$,
	\item $f$ má derivaci v každém bodě intervalu $(a,b)$.
\end{itemize}
\vspace{-5pt}
Potom existuje $c\in(a,b)$ tak, že $f'(c)=\frac{f(b)-f(a)}{b-a}$.

\theorem Buď $f$ funkce spojitá na intervalu $J$, která má druhou derivaci v každém bodě intervalu $J^{\circ}$ (buď $J$ interval s krajnými body $a$ a $b$, potom $J^{\circ}=(a,b)$ ).
\vspace{-5pt}
\begin{itemize}
	\item Funkce $f$ je konvexní na intervalu $J$, právě když $f''(x) \geq 0$ pro každé $x \in J^{\circ}$.
	\item Je-li $f''(x) > 0$ v každém bodě $x \in J^{\circ}$, pak je $f$ ryze konvexní na $J$.
	\item Funkce $f$ je konkávní na intervalu $J$, právě když $f''(x) \leq 0$ pro každé $x \in J^{\circ}$.
	\item Je-li $f''(x) < 0$ v každém bodě $x \in J^{\circ}$, pak je $f$ ryze konkávní na $J$.
\end{itemize}
\vspace{-5pt}


% =========================================================
\section{35 -- Základy integrálního počtu (primitivní funkce, neurčitý integrál, Riemannův integrál (definice, vlastnosti a geometrický význam)).}

\definition Nechť $f$ je funkce definovaná na intervalu $(a, b)$, kde $a,b\in\overline{\mathbb{R}}$. Funkci $F$ splňující podmínku
$$
	F' (x) = f (x) \mathrm{pro každé} x \in (a, b)
$$
nazýváme \textbf{primitivní funkcí} k funkci $f$ na intervalu $(a, b)$.

\theorem Nechť $F$ je primitivní funkcí k funkci $f$ na intervalu $(a, b)$. Pak $G$ je primitivní funkcí k funkci $f$ na intervalu $(a, b)$ právě tehdy, když existuje konstanta $c \in \mathbb{R}$ taková, že
$$
	G(x) = F (x) + c, \mathrm{pro každé} x \in (a, b).
$$

\definition Nechť k funkci $f$ existuje primitivní funkce na intervalu $(a, b)$. Množinu všech primitivních funkcí k funkci $f$ na $(a, b)$ nazýváme \textbf{neurčitým integrálem} a značíme jej $\int f (x) \mathrm{d}x$.

\theorem[Postačující podmínka pro existenci primitivní funkce] Nechť funkce $f$ je spojitá na intervalu $(a, b)$. Pak funkce $f$ má na tomto intervalu primitivní funkci.

\theorem Nechť $F$, resp. $G$, je primitivní funkce k funkci $f$, resp. $g$, na intervalu $(a, b)$ a nechť $\alpha\in\mathbb{R}$. Pak
\vspace{-5pt}
\begin{itemize}
	\item $F + G$ je primitivní funkcí k funkci $f + g$ na intervalu $(a, b)$,
	\item $\alpha F$ je primitivní funkcí k funkci $\alpha f$ na intervalu $(a, b)$.
\end{itemize}
\vspace{-5pt}

\theorem [Per partes] Nechť funkce $f$ je diferencovatelná na intervalu $(a, b)$ a $G$ je primitivní funkce k funkci $g$ na intervalu $(a, b)$ a konečně nechť existuje primitivní funkce k funkci $f' G$. Potom existuje primitivní funkce k funkci $fg$ a platí
$$
	\int fg = fG - \int f'G.
$$

\theorem [Substituce I] Nechť pro funkce $f$ a $\varphi$ platí
\vspace{-5pt}
\begin{enumerate}
	\item $f$ má primitivní funkci $F$ na intervalu $(a, b)$,
	\item $\varphi$ je na intervalu $(\alpha, \beta)$ diferencovatelná,
	\item $\varphi((\alpha, \beta)) \subset (a, b)$.
\end{enumerate}
\vspace{-5pt}
Pak funkce $f(\varphi(x)) \cdot \varphi'(x)$ má primitivní funkci na intervalu $(\alpha, \beta)$ a platí
$$
	\int f(\varphi(x))\cdot \varphi'(x)\mathrm{d}x = F(\varphi(x)).
$$

\theorem [Substituce II] Nechť $f$ je definována na intervalu $(a, b)$ a nechť $\varphi$ je bijekce intervalu $(\alpha, \beta)$ na $(a, b)$ s nenulovou konečnou derivací. Pak platí
$$
	\int f(\varphi(t))\varphi'(t)\mathrm{d}t=G(t)+C \quad\Rightarrow\quad \int f(x)\mathrm{d}x=G(\varphi^{-1}(x))+C
$$

% ---------------------------------------------------------
\subsection{Riemannův integrál}
\textit{Definica je dlhá, viď skripta. Vlastnosti sú také, ako očákavaš.}

\theorem[Newtonova formule] Nechť $f$ je funkce spojitá na intervalu $\langle a,b\rangle$ s primitivní funkcí $F$. Pak platí rovnost
$$
	\int_a^b f(x)\mathrm{d}x = F(b) - F(a) =: \big[ F(x) \big]_a^b
$$

% =========================================================
\section{36 -- Číselné řady (konvergence číselné řady, kritéria konvergence, odhadování rychlosti růstu řad pomocí určitého integrálu).}

\definition[Číselná řada]\label{def:rada} Formální výraz tvaru
$$
	\sum_{k=0}^{\infty}a_k=a_0+a_1+\dots,
$$
kde $(a_k)_{k=0}^{\infty}$ je zadaná \hyperref[def:posloupnost]{číselná posloupnost}, nazýváme \textbf{číselnou řadou}. Pokud je posloupnost částečných součtů
$$
	s_n:=\sum_{k=0}^n a_k,\quad n\in\mathbb{N}_0,
$$
\hyperref[def:konvergence]{konvergentní}, nazýváme příslušnou řadu také \textbf{konvergentní}. V opačném případě mluvíme o divergentní číselné řadě. \textbf{Součtem} konvergentní řady $\sum_{k=0}^n a_k$ nazýváme hodnotu limity $\lim\limits_{n\rightarrow\infty} s_n$.

% ---------------------------------------------------------
\subsection{Kritéria konvergence}

\theorem[Nutná podmínka konvergence]\label{th:nutna_konvergence} Pokud řada $\sum_{k=0}^{\infty}a_k$ konverguje, potom pro limitu jejích sčítanců platí $\lim\limits_{k\rightarrow\infty} a_k = 0$. \hspace{.5cm}$\Rightarrow$\hspace{.5cm} Pokud limita posloupnosti $(a_k)_{k=0}$ je nenulová nebo neexistuje, potom řada $\sum_{k=0}^{\infty} a_k$ není konvergentní.

\theorem[Bolzano-Cauchy] Řada $\sum_{k=0}^{\infty} a_k$ konverguje právě tehdy, když pro každé $\varepsilon >0$ existuje $n_0\in\mathbb{R}$ tak, že pro každé $n \geq n_0$ a $p\in\mathbb{N}$ platí
$$
	|a_n+a_{n+1}+\dots+a_{n+p}|<\varepsilon.
$$

\definition[Absolutní konvergentnost] \label{def:abs_konv} Číselnou řadu $\sum_{k=0}^{\infty} a_k$ nazýváme \textbf{absolutně konvergentní}, pokud číselná řada $\sum_{k=0}^{\infty} |a_k|$ konverguje.

\theorem Pokud řada \hyperref[def:abs_konv]{absolutně konverguje}, potom tato řada \hyperref[def:rada]{konverguje}.

\theorem[Leibnizovo kritérium] Buď $(a_k)_{k=0}^{\infty}$ klesající posloupnost s nezápornými členy konvergujíci k nule. Potom je řada
$$
	\sum_{k=0}^{\infty}(-1)^k a_k
$$
konvergentní. Toto kritérium platí i pokud je $(a_k)_{k=0}^{\infty}$ rostoucí posloupnost záporných čísel konvergující k nule.

\theorem [Srovnávací kritérium] Buďte $\sum_{k=0}^{\infty} a_k$ a $\sum_{k=0}^{\infty} b_k$ číselné řady. Potom platí následující dvě tvrzení.
\vspace{-5pt}
\begin{enumerate}
	\item Nechť pro každé $k\in\mathbb{N}$ platí nerovnosti $0\leq |a_k|\leq b_k$ a nechť řada $\sum_{k=0}^{\infty} b_k$ konverguje. Potom řada $\sum_{k=0}^{\infty} a_k$ absolutně konverguje.
	\item Nechť pro každé $k\in\mathbb{N}$ platí nerovnosti $0\leq a_k\leq b_k$ a $\sum_{k=0}^{\infty} a_k$ diverguje. Potom i řada $\sum_{k=0}^{\infty} b_k$ diverguje.
\end{enumerate}
\vspace{-5pt}

\theorem [d'Alembertovo kritérium] Nechť $a_k > 0$ pro každé $k \in\mathbb{N}_0$ . Pokud
$$
	\lim_{k\rightarrow\infty}\frac{a_k+1}{a_k}>1,
$$
potom řada $\sum_{k=0}^{\infty} a_k$ diverguje. Pokud ovšem
$$
	\lim_{k\rightarrow\infty}\frac{a_k+1}{a_k}<1,
$$
potom řada $\sum_{k=0}^{\infty} a_k$ konverguje.

% \clearpage

% V tejto otázke budeme hľadať čo najkratšie cesty v stavovom priestore s ohodnotenými akciami. Tento priestor si môžeme predstaviť ako neorientovaný graf s ohodnotenými hranami. 
% \begin{definition}[Stavový priestor s ohodnotenými akciami]
% Nech $(S,A)$ je stavový priestor, kde $S$ je množina javov a $A$ je množina akcií. Trojicu $(S,A,c)$, kde $c$ je funkcia $A\rightarrow\mathbb{R}_0^+$, nazývame stavový priestor s ohodnotenými akciami. Hodnota $c(a)$ priraďuje cenu každej akcií $a\in A$.
% \end{definition}

% Všetky spomenuté algoritmy pracujú s expanziou stavu. Pri expanzií pridáme do poolu možných nasledujúcich stavov $OPEN$, všetkých susedov expanzovaného stavu, ktorých sme ešte neobjavili.

% % =========================================================
% \section{Dijkstrov algoritmus}
% Dijkstrov algoritmus nájde najkratšiu cestu z počiatočného stavu do všetkých ostatných v čase $\mathcal{O}(|A|+|S|\cdot\log(|S|))$.

% Myšlienka algoritmu spočíva v udržovaní si najkratšej známej cesty $g(s)$ z počiatočného vrcholu do vrcholu $s$ pre všetky vrcholy $s\in OPEN$, ktoré sme už objavili. Vždy sa expanduje stav $s^*\in OPEN$ s najkratšou cestou z počiatočného stavu
% $$
% 	s^*\in\arg\min_{s\in OPEN}g(s).
% $$
% Pokiaľ sme expanziou našli kratšiu cestu do nejakého stavu $s\in OPEN$, dĺžku cesty upravíme. Rozmyslite si, že v momente expanzie stavu už nemôže existovať kratšia cesta do tohto stavu ako tá, ktorú poznáme. Vizualizácia behu algoritmu sa dá nájsť na \url{https://www.cs.usfca.edu/~galles/visualization/Dijkstra.html}.

% % =========================================================
% \section{Informované prehľadavánie stavového priestoru}
% Dijkstra je ako BFS berúce odhad na dĺžku cesty od počiatočnéhos tavu. Nájde síce optimálne riešenie, no expanduje príliš veľa vrcholov na všetky strany a teda je v praxi pomalý. Jeden so spôsobov ako expandovať vrcholy inteligentnejšie smerom ku cielu je \textbf{heuristika}.

% \begin{definition}[Heuristika]
% 	Nech $(S,A,c)$ je stavový priestor s ohodnotenými hranami a $G\subseteq S$ je množina koncových stavov pre úlohu prehľadávania tohto priestoru. \textbf{Heuristika} $h$ je ľubovoľná funkcia
% 	$$h: S\rightarrow \mathbb{R}_0^+$$
% 	taká, že $h(s)$ udáva odhad ceny cesty zo stavu $s\in S$ ku najbližšiemu $s_g\in G$ a pre všetky $s_g \in G$ platí $h(s_g)=0$.
% \end{definition}

% % -----------------------------------------------
% \subsection{Greedy search}
% Algoritmus, ktorý v každom kroku expanduje vrchol $s^*$ s najmenšou hodnotou heuristickej funkcie (teoreticky najbližšie ku cielu):
% $$ s^*\in\arg\min_{s\in OPEN}h(s).$$

% Greedy search má niekoľko problémov:
% \begin{itemize}
% 	\item algoritmus nenájde najkratšiu cestu (Fig. \ref{fig:greedy})
% 	\item rýchlosť a kvalita výsledku závisí od rýchlosti výpočtu a kvality heuristiky
% \end{itemize}
% \begin{figure}
% 	\centering
% 	\includegraphics[width=.5\linewidth]{fig/greedy.png}
% 	\caption{Greedy search nenájde najkratšiu cestu.}
% 	\label{fig:greedy}
% \end{figure}

% % -----------------------------------------------
% \subsection{Algoritmus A*}
% Algoritmus A* kombinuje oba zmienené postupy, čím sa snaží eliminovať problémy. Označme $g(s)$ cenu cesty od počiatočného vrcholu vypočítanú pomocou Dijkstru a $h(s)$ hodnotu heuristiky. Potom A* expanduje vrchol $s^*$, ktorý minimalizuje súčet týchto dvoch vzdialeností:
% $$ s^*\in\arg\min_{s\in OPEN}\big ( h(s) + g(s) \big ).$$

% Vizualizácia sa nachádza na \url{http://www.jsgl.org/doku.php?id=pathfinder}.

% % -----------------------------------------------
% \subsection{Štúdium heuristík}
% \begin{definition}[Optimálna heuristika]
% 	\textbf{Optimálnou heuristikou $h^*$} nazývame takú heuristiku, ktorá vracia skutočnú cenu cesty do cieľového stavu. V praxi je z výpočetných dôvodov nesplniteľným snom -- vieme ju vypočítať hrubou silou, to nám ale algoritmus používajúci túto heuristiku nezrýchli. Má dôležktý teoretický význam -- lepšia heuristika sa nedá skonštruovať.
% \end{definition}

% \begin{definition}[Prípustná heuristika]
% 	Nech $(S,A,c)$ je stavový priestor s ohodnotenými hranami a $h$ je heuristika. Heuristiku $h$ nazývame \textbf{prípustnou} pokiaľ
% 	$$ \forall s \in S:h(s)\leq h^*(s). $$
% \end{definition}

% \begin{definition}[Monotónna heuristika]
% 	Nech $(S,A,c)$ je stavový priestor s ohodnotenými hranami a $h$ je heuristika. Heuristiku $h$ nazývame \textbf{monotónnou} pokiaľ
% 	$$ \forall x,y \in S, \forall (x,y) \in A:h(x)\leq h(y) + c(x,y). $$
% 	Neformálne, heuristika stavu $s\in S$ je vždy menšia rovná heuristike každého zo susedných stavou plus cena cesty do daného stavu.
% \end{definition}

% \begin{definition}[Dominujúca heuristika]
% 	Nech $(S,A,c)$ je stavový priestor s ohodnotenými hranami a $h_1$ a $h_2$ sú \textit{prípustné} heuristiky. Heuristika $h_1$ \textbf{dominuje} heuristiku $h_2$ pokiaľ
% 	$$ \forall s \in S:h_1(s)\leq h_2(s). $$
% 	Heuristika $h^*$ dominuje všetky prípustné heuristiky. Neformálne, ak heuristika $h_1$ dominuje heuristiku $h_2$, znamená to, že $h_1$ lepšie odhaduje optimálnu heuristiku.
% \end{definition}


% \printbibliography

\end{document}
